\documentclass[11pt]{article}
\usepackage{sectsty}
\usepackage{enumerate}
\usepackage{bm}
\usepackage{amsmath, amsthm, amssymb}
\usepackage[usenames,dvipsnames]{color}
\usepackage{float, graphicx}
\usepackage{environ}
\usepackage{booktabs, xeCJK}

\providecommand{\abs}[1]{\lvert#1\rvert}
\providecommand{\norm}[1]{\lVert#1\rVert}

\usepackage[sc]{mathpazo} % math & rm
\linespread{1.05}        % Palatino needs more leading (space between lines)
\usepackage[scaled=0.90]{helvet} % ss
% \usepackage[scaled=0.85]{sourcecodepro} % tt
\usepackage[T1]{fontenc}
\usepackage{textcomp}


\newtheorem{thm}{Theorem}
\newtheorem{lemma}[thm]{Lemma}
\newtheorem{fact}[thm]{Fact}
\newtheorem{cor}[thm]{Corollary}
\newtheorem{eg}{Example}
\newtheorem{ex}{Exercise}
\newtheorem{defi}{Definition}
\newtheorem{hw}{Problem}
\newenvironment{sol}
{\par\vspace{3mm}\noindent{\it Solution}.}
{\qed}

\newcommand{\ov}{\overline}
\newcommand{\cb}{{\cal B}}
\newcommand{\cc}{{\cal C}}
\newcommand{\cd}{{\cal D}}
\newcommand{\ce}{{\cal E}}
\newcommand{\cf}{{\cal F}}
\newcommand{\ch}{{\cal H}}
\newcommand{\cl}{{\cal L}}
\newcommand{\cm}{{\cal M}}
\newcommand{\cp}{{\cal P}}
\newcommand{\cz}{{\cal Z}}
\newcommand{\eps}{\varepsilon}
\newcommand{\ra}{\rightarrow}
\newcommand{\la}{\leftarrow}
\newcommand{\Ra}{\Rightarrow}
\newcommand{\dist}{\mbox{\rm dist}}
\newcommand{\bn}{{\mathbf N}}

\newcommand{\bA}{ \bm{A} }
\newcommand{\bI}{ \bm{I} }
\newcommand{\bM}{ \bm{M} }
\newcommand{\bx}{ \bm{x} }
\newcommand{\bV}{ \bm{V} }
\newcommand{\bC}{ \bm{C} }
\newcommand{\bc}{ \bm{c} }
\newcommand{\bU}{ \bm{U} }
\newcommand{\bR}{ \bm{R} }
\newcommand{\bW}{ \bm{W} }
\newcommand{\bX}{ \bm{X} }
\newcommand{\bY}{ \bm{Y} }
\newcommand{\bZ}{ \bm{Z} }
% \setlength{\parindent}{0pt}
% \setlength{\parskip}{2ex}
% \newenvironment{proofof}[1]{\bigskip\noindent{\itshape #1. }}{\hfill$\Box$\medskip}

\usepackage{enumerate,fullpage,proof,color,hyperref}


\newcommand{\todo}[1] { \color{red}[TODO: #1]\color{black} }


\newcommand{\alns}[1] {
	\begin{align*} #1 \end{align*}
}
\newcommand{\pd}[2] {
  \frac{\partial #1}{\partial #2}
}

\title{Big Data Processing: homework 3}
\author{凌康伟 \qquad 5140219295}

\begin{document}
\maketitle

\subsection*{Exercise 6.1.1}
\begin{enumerate}[(a)]
\item In this setting, for items numbered beyond 20, there are less than 5
  multiples for each number, therefore they are not frequent. The frequent items
  are: 1, 2, ..., 20;
\end{enumerate}

\subsection*{Exercise 6.1.5}
\begin{enumerate}[(a)]
\item The baskets that contains item 5 and 7 are 35, 70 (common multiples of 5
  and 7), thus support($\{5, 7\}$) = 2. Since 70 is also a multiple of 2, item 2
  is in basket 70, support($\{2, 5, 7\}$) = 1. Therefore, conf($\{5, 7\}
  \rightarrow 2$) = $1/2$.
\item The baskets containing item 2, 3 and 4 are 12, 24, 36, 48, 60, 72, 84, 96,
  and among these basket, 60 contains 5. Therefore,
  \begin{align*}
    \text{support}(\{2, 3, 4\}) &= 8 \\
    \text{support}(\{2, 3, 4, 5\}) &= 1 \\
    \text{conf}(\{2, 3, 4\} \rightarrow 5) &= \frac{1}{8}
  \end{align*}
\end{enumerate}

\subsection*{Exercise 11.1.3}
Let $A$ denote any symmetric 3x3 matrix:
\[
  \bA =
  \begin{bmatrix}
    a & b & c\\
    b & d & e\\
    c & e & f
  \end{bmatrix}
\]
And for $\bA - \lambda\bI $:
\begin{align*}
  \abs{\bA - \lambda\bI} =
  \begin{vmatrix}
   a-\lambda & b & c\\
    b & d-\lambda & e\\
    c & e & f-\lambda
  \end{vmatrix} &= (a-\lambda)[(d-\lambda)(f-\lambda) - e^2] - b[b(f-\lambda)- ce] + c[be - c(d-\lambda)] \\
             &= -\lambda^3 + (a+d+f)\lambda^2 - (df + ad + af- e^2- b^2-c^2)\lambda \\
                 &\qquad + adf + 2bce - ae^2 - fb^2 - dc^2
\end{align*}
Therefore the equation can be expressed as:
\[
  \alpha_3\lambda^3 + \alpha_2\lambda^2 + \alpha_1\lambda + \alpha_0 = 0
\]
where
\begin{align*}
  \alpha_3 &= {-1} \\
  \alpha_2 &= a + d + f \\
  \alpha_1 &= - (df + ad + af - e^2 - b^2 - c^2) \\
  \alpha_0 &= adf + 2bce - ae^2 - fb^2 -dc^2
\end{align*}

\subsection*{Exercise 11.2.1}
\begin{enumerate}[(a)]
\item $\bM^T\bM =
  \begin{bmatrix}
    30 & 100 \\
    100 & 354 
  \end{bmatrix}
$, $\bM\bM^T =
\begin{bmatrix}
  2 & 6 & 12 & 20 \\
  6 & 20 & 42 & 72 \\
  12 & 42 & 90 & 156 \\
  20 & 72 & 156 & 272
\end{bmatrix}
$
\item First sovle the equation $\abs{\bM^T\bM - \lambda I} = 0$ for eigenvalues:
  \[
    \begin{vmatrix}
      30 - \lambda & 100 \\
      100 & 354 - \lambda
    \end{vmatrix}
    = (30 -\lambda)(354-\lambda) - 10000 = \lambda^2 - 384\lambda + 620 = 0
  \]
  $\lambda_1 \approx 1.6214, \lambda_2 \approx 382.3786$.

  Second find the eigenvectors associated with each eigenvalue:
  \begin{enumerate}[i.]
  \item $\lambda_1 = 1.6214$. for the following equation:
    \[
      \begin{bmatrix}
        28.3786 & 100 \\
        100 &  352.3786
      \end{bmatrix}
       \bx = 0
     \]
     a normalized solution: $\bx_1 =
     \begin{bmatrix}
       0.9620 & 0.2730
     \end{bmatrix}^T
$
\item $\lambda_2 = 382.3786$, eigenvector $\bx_2 =
  \begin{bmatrix}
    0.2730 & 0.9620
  \end{bmatrix}^T
  $
  \end{enumerate}
\item The same eigenvalues as $\bM^T\bM$, $\lambda_1 \approx 1.6214, \lambda_2
  \approx 382.3786$, and $\lambda_3 = \lambda_4 = 0$
  
\item
  \begin{align*}
\bx_1 &=
  \begin{bmatrix}
    0.5425 & 0.6527 & 0.3364 & {-0.4079}
  \end{bmatrix}^T \\
  \bx_2 &=
  \begin{bmatrix}
    0.0631 & 0.2247 & 0.4846 & 0.8429
  \end{bmatrix}^T \\
  \bx_3 &=
  \begin{bmatrix}
    0.6882 & {-0.6882} & 0.2294 & 0
  \end{bmatrix}^T \\
  \bx_4 &=
  \begin{bmatrix}
    0.7960 & {-0.5970} & 0 & {0.0995}
  \end{bmatrix}^T
\end{align*}
\end{enumerate}

\subsection*{Exercise 11.3.2}
\[
  q = [0, 3, 0, 0, 4] \times \bV = [1.74 \quad 2.84]
\]
So Leslie is high in romantic interest, and a little bit interested in sci-fi.
Use $q$ to predict Leslie's rating for other movies.
\[
  r = q \bV^T =
  \begin{bmatrix}
    1.009 & 1.009 & 1.009 & 2.016 & 2.016
  \end{bmatrix}
\]
So it suggests that Leslie may like romantic movies(Titanic, Casablanca) very
much, and also have some interests for other sci-fi movies.

\subsection*{Exercise 11.4.2}
\begin{enumerate}[(a)]
\item The square of the Frobenius norm of $\bM$ is $f_{\bM} = 4(1 + 9 + 16 + 25)
  + 3 (16 + 25 + 4) = 243$.

  The columns for \textit{The Matrix} and \textit{Alien} each have a squared
  Frobenium norm of 51, therefore their probability is $51/243 = 0.210$. The scaler is $\sqrt{2P(j)} = 0.648$.
  \[
    \bC =
    \begin{bmatrix}
      \dfrac{\bc_{\mathit{matrix}}}{\sqrt{2P(\mathit{matrix})}}
      & \dfrac{\bc_{\mathit{alien}}}{\sqrt{2P(\mathit{alien})}}
    \end{bmatrix}
    =
    \begin{bmatrix}
      1.54 & 4.63 & 6.17 & 7.72 & 0 & 0 & 0 \\
      1.54 & 4.63 & 6.17 & 7.72 & 0 & 0 & 0
    \end{bmatrix}^T
  \]

  The rows for Jim and John each have a squared Frobenium norm of 27, 48,
  respectively, therefore the probabilities are $27/243 = 0.111, 48/243=0.198$,
  respectively. The scalers are $\sqrt{2P(Jim)} = 0.471, \sqrt{2P(John)} =
  0.629$.
  \[
    \bR =
    \begin{bmatrix}
      \dfrac{\bc_{Jim}}{\sqrt{2P(Jim)}} \\
      \dfrac{\bc_{John}}{\sqrt{2P(John)}}
    \end{bmatrix}
    =
    \begin{bmatrix}
      6.37 & 6.37 & 6.37 & 0 & 0 \\
      6.36 & 6.36 & 6.36 & 0 & 0 
    \end{bmatrix}
  \]
  $\bW =
  \begin{bmatrix}
    3 & 3 \\
    4 & 4
  \end{bmatrix}$. Use svd on $\bW$:
  \begin{align*}
    \bW = \bX\bZ\bY^T &=
    \begin{bmatrix}
      0.6 & {-0.8}\\
      0.8 & 0.6
    \end{bmatrix}
    \cdot
    \begin{bmatrix}
      \sqrt{50} & 0 \\
      0 & 0
    \end{bmatrix}
    \cdot
    \begin{bmatrix}
      0.707 & 0.707 \\
      0.707 & {-0.707}
    \end{bmatrix} \\
    \bZ^+ &= \begin{bmatrix}
      \dfrac{1}{\sqrt{50}} & 0 \\
      0 & 0
    \end{bmatrix}
  \end{align*}
  Then,
  \[
    \bU = \bY (\bZ^+)^2\bX^T =
    \begin{bmatrix}
      0.0085 & 0.0113 \\
      0.0085 & 0.0113 
    \end{bmatrix}
  \]
  Therefore, the CUR decomposition of $\bM$ is:
  \[
    \bM \approx \bC \cdot \bU \cdot \bR
    = \begin{bmatrix}
      1.54 & 4.63 & 6.17 & 7.72 & 0 & 0 & 0 \\
      1.54 & 4.63 & 6.17 & 7.72 & 0 & 0 & 0
    \end{bmatrix}^T \cdot
    \begin{bmatrix}
      0.0085 & 0.0113 \\
      0.0085 & 0.0113 
    \end{bmatrix} \cdot
    \begin{bmatrix}
      6.37 & 6.37 & 6.37 & 0 & 0 \\
      6.36 & 6.36 & 6.36 & 0 & 0 
    \end{bmatrix}
    \]
\end{enumerate}
\end{document}